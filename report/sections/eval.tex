\chapter{Critical evaluation of the prototype submitted}\label{ch:eval}\vspace{-10mm}

\section{Strengths}
The greatest strength of our 'Mathboard' application is its feature richness while still maintaining total ease of use, "viewers" can begin learning in just seconds with only a room number, this is very important for a learning environment that is often under time constraints. "Teachers" need no prior knowledge to start using the 'Mathboard' right away. Other strengths of this application is the chat functionality, this allows both interactive learning, communicating with the teacher, and collaborative learning, by sharing ideas with the other "viewers" of the session. One of the greatest strengths of this application is the immediate updates from the "host" to the "viewers" with only a one second refresh rate for an unlimited number of users.

\section{Weaknesses}
Upon testing of the application, we have noticed a weaknesses that may require further development, there is no easy method implemented within the 'Mathboard' to notify the "viewers" of which room number they should be entering, this would therefore have to be transferred via alternate communication streams.

\section{Time Management}
Agile development was a great technique to ensure we were never overwhelmed by the quantity of work left to produce with this project. The use of frequent scrums and pair programming set a good structure for completing tasks although they were not stuck to as definite rules.

\section{Further Development}
At the top of the next-features-to-add list would have been to have some form of public room as well as private. These would be listed on the front page with a brief description for public class rooms. Other features that we would have liked to have implemented:
\begin{itemize}
   \item Hotkeys for the difference tools when drawing on the Mathboard (e.g. 'r' brings up red pen).\vspace{-2mm}
   \item Further editing of existing objects (e.g. change the colour of a line that has already been drawn).\vspace{-2mm}
   \item Bigger whiteboard, and crucially one that scrolls. As it is a small screened device has a limited space to work within.\vspace{-2mm}
   \item Multi-user editing. It would have been desirable to have Mathboards that are universally editable, and also an option in a teacher-student Mathboard to give a student temporary access to 'write on the board'.\vspace{-2mm}
   \item Maths typesetting in the chat window so that students could communicate complicated mathematical expressions in the same way that the teacher can through the board.\vspace{-2mm}
   \item This item would likely have never been possibly with the limitations of GAE (when used for free), but complimentary audio and or video streams alongside the board, or even layered, would have made the tool even more versatile (if less lightweight).\vspace{-2mm}
\end{itemize}