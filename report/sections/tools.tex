\chapter{List of tools and techniques used}\label{ch:tools}\vspace{-10mm}
We used many of the agile development processes' standard tools and structure. Specifically we made use of:

\section{Git (GitHub)}
One of the most important tools to the completion of this project was Git. We used a single, private repository (shared with all members of the group) initially only in order to track and maintain progress as a group.

This was then forked by different teams working on different parts of the project. These elements were brought together throughout using pull requests.

Extensive use was also made of the inbuilt GitHub editor for small changes - saving significant time over pulling, adding, committing and pushing on a local machine.

\section{The Google App Engine (GAE) SDK}
Although this partly goes without mentioning for a GAE based project, there were two elements worth comment. Firstly, the local development testing environment which allowed the running of a GAE application on the localhost. This greatly sped up the development process as the deploy process (to Google's appspot service) was fairly slow (about a minute).

For the most part the application was deployed only by those working on the back-end of the application. For ease we had two development GAE projects for different members to use, since the process of sharing a project was found to be generally unhelpful.

Within the platform, we used the Python27 execution environment for serving the graphics data and HTML pages, and the Google Datastore for persistence.

\section{Pair Programming}
Since the project team was fairly large (6 members), it was decided pair-programming should be used where possible. This was thought particularly important in-case the project schedule spilled into the Christmas vacation with the danger of team members being unreachable.
As such, a broad division of work was decided on with:
\vspace{-3mm}
\begin{itemize}
  \item Michael and Simeon working on the python back-end
  \vspace{-3mm}
  \item James and Dan working on the lobby system
  \vspace{-3mm}
  \item Pip and Josh working on the front end JavaScript
\end{itemize}

This was only an outline and was followed less in the latter stages of development but provided a framework to start work and ensure that there were always people available to explain any part of the project.

\section{Scrum Meetings}
Particularly at the beginning of the project, when the broad-stroke ideas where being put in place, we held (approximately) weekly meetings to discuss progress and aims.

\begin{flushright}
These were only some of the methods used - although they were the most important.
\end{flushright}